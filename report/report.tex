\documentclass{article}

\usepackage[utf8]{inputenc}
\usepackage[french]{babel}

\author{
    Valentin Jonquière,
    Mathilde Chollon
}

\title{Rapport TD1 OS}

\begin{document}

\maketitle

\pagebreak

\tableofcontents

\pagebreak

\section{Bilan}


\section{Points Délicats}


\section{Limitations}
Au niveau des fuites mémoires, nous n'avons pas trouvé de mémoire perdue En revanche, nous nous sommes rendus compte qu'il y avait toujours de la mémoire accessible après la fin de l'exécution du programme.
Nous avons donc testé les mêmes commandes de détection de fuite mémoire sur notre code actuel et sur le code qui nous a été fourni au début du projet. Sur les tests que nous n'avions pas implémenté, il y avait tout de même
de la mémoire encore accessible. Nous en avons donc déduit que c'était du au thread principal, que nous ne pouvons donc pas détruire étant donné qu'il est la source de la machine.


\section{Tests}

\section{Brouillon}
\subsection{Partie I}
\begin{itemize}
    \item on s'attend à voir quatre caractères, 'abcd' car on peut voir une boucle qui ajoute 1 au caratère donné (a)

    
\end{itemize}

\subsection{Partie II}
\begin{itemize}
    \item C'est une erreur de lire avant d'être averti pour plusieurs raisons. Il n'y a peut être pas de charactère tapé, et on ne peut pas lire un caractère nul puisqu'il n'existe pas.
    L'écriture du caractère précédant pourrait ne pas être finie, et dans ce cas là, on pourrait avoir la lecture d'un caractère qui n'est pas le notre.
    De plus, on peut faire des combo de touches pour taper un caratère, si on lit avant que l'on ait finit, on se retrouvera avec le mauvais caractère.

    
\end{itemize}


\subsection{Partie VI}
\begin{itemize}
    \item Message d'erreur `Unimplemented system call 1`, car l'appel SC\_Exit n'est pas implémenté. On se sert de SC\_Halt pour arrêter la machine.
    On peut dupliquer l'effet de SC\_Halt dans SC\_Exit afin de ne plus à avoir à l'appeler.
    \item Changement de registre dans Start.s pour recupérer la valeur de retour
    \item Que faire de la valeur de retour ??
\end{itemize}

\end{document}