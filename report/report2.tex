\documentclass{article}

\usepackage[utf8]{inputenc}
\usepackage[french]{babel}

\author{
    Valentin Jonquière,
    Mathilde Chollon
}

\title{Rapport TD2 OS}

\begin{document}

\maketitle

\pagebreak

\tableofcontents

\pagebreak

\section{Bilan}

\section{Points Délicats}
Nous avons passé du temps sur la création de mutex. En effet, nous avons commencé par les créer à l'aide
des sémaphores initialisés à un afin de ne pas copier/coller la quasi-totalité de leurs fonctions. Après
une discussion avec notre chargé de TD, nous avons décidé d'implémenter les mutex sur la base des sémaphores,
en mettant un booléen \textit{locked} pour représenter l'état du mutex à la place de l'entier \textit{value} ainsi
qu'en rajoutant un pointeur vers le thread ayant verrouillé le mutex.

Lorsque nous avons voulu remonter l'utilisation des sémaphores vers l'espace utilisateur, nous avons beaucoup réfléchi à l'implémentation.
Nous avons rajouté quatre appels systèmes : \textit{SemaphoreCreate}, \textit{SemaphoreDelete}, \textit{P} et \textit{V}. Nous déléguons aux
threads d'un même espace d'adressage ont accès aux mêmes sémaphores. Pour cela, nous avons un \textit{Bitmap} afin de savoir quels sémaphores sont 
créés et un tableau de Sémaphores pour les stocker. L'utilisateur n'a réellement accès qu'à l'indice de son sémaphore.

\section{Limitations}
Nous utilisons la variable \textit{UserStacksAreaSize} du fichier \textit{addrspace.h} afin de créer les piles de nos threads. Nous avons gardé la valeur de départ 1024, 
ce qui nous donne quatre piles de 256 octets. L'utilisateur ne peut donc avoir que 4 threads en simultané. Si nous avons besoin de plus, il suffit de modifier la variable. 
Nous avons gardé cette valeur, car il était plus simple de se rendre compte de possibles erreurs avec 4 threads.

Beaucoup de memory leak avec les threads : les fonctions de Cleanup ne suppriment pas la stack.
En effet, il y a bien une fonction \textit{StackAllocate} mais pas de \textit{StackDeallocate}.
Nous nous retrouvons donc avec 73K octets non supprimés à la fin de l'exécution lorsque nous utilisons des threads.

Concernant les sémaphores utilisateurs, nous avons choisi d'avoir un bitmap de taille 16, donc 16 sémaphores en simultané pour un même espace d'adressage.
Cette limite est modifiable dans le fichier \textit{addrspace.cc} à \textit{MAX_SEMAPHORES}, dans le cas où 16 serait une limite bien trop faible.


\section{Tests}

\section{Brouillon}


\end{document}