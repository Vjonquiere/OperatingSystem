\documentclass{article}

\usepackage[utf8]{inputenc}
\usepackage[french]{babel}

\author{
    Valentin Jonquière,
    Mathilde Chollon
}

\title{Rapport TD2 OS}

\begin{document}

\maketitle

\pagebreak

\tableofcontents

\pagebreak

\section{Bilan}
Toutes les actions semblent fonctionner dans l'ensemble, mais nous avons rencontré 
un problème avec l'utilisation de \textit{printf}. En effet, cette fonction ne gère pas 
correctement les boucles \textit{for}, ce qui rend son utilisation compliquée dans notre
projet. Pour contourner cette limitation, nous avons utilisé \textit{PutString} et 
\textit{PutInt} à la place de \textit{printf} dans nos tests et nos implémentations, 
ce qui a permis d'éviter des erreurs inattendues lors de l'affichage. Nous avons tenté de le 
modifier en le sécurisant avec des sémaphores, mais sans succès. Notre code de \textit{vsprintf.c}
se trouve dans l'espace utilisateur et ne comporte pas de main où créer le sémaphore comme dans
nos programmes de tests. Nous ne pouvions pas non plus faire comme pour \textit{PutString}
car nous ne sommes pas en espace noyau. La partie I ne nous a pas particulièrement posé de problèmes, nous
ne nous sommes pas attardés dessus dans le rapport.

\section{Points Délicats}
\subsection{Partie II}
Nous avons passé du temps sur la création de mutex. En effet, nous avons
commencé par les créer à l'aide des sémaphores initialisés à un afin de ne pas
copier/coller la quasi-totalité de leurs fonctions. Après une discussion avec
notre chargé de TD, nous avons décidé d'implémenter les mutex sur la même base
de code que les sémaphores, en mettant un booléen \textit{locked} pour
représenter l'état du mutex à la place de l'entier \textit{value} ainsi qu'en
rajoutant un pointeur vers le thread ayant verrouillé le mutex.

Nous avons ensuite utilisé ces mutex afin de sécuriser \textit{PutChar} et \textit{GetChar}.
Pour cela, nous avons utilisé deux mutex, \textit{writeLock} et \textit{readLock}. Il fallait utiliser
deux mutex car nous pouvons faire \textit{PutChar(GetChar)} par exemple, ce qui ne fonctionnerait
pas avec seulement un mutex. Nous avons également protégé \textit{PutString} et \textit{GetString}.
En effet, sans protection, nous pouvions être interrompus lors d'un changement de contexte par un autre
thread utilisant également \textit{PutString}, ce qui entremêlerait les deux strings.


Avant de faire l'\textit{Action II.3}, si nous lancions plusieurs threads, ils avaient la même pile.
Nous ne pouvions pas allouer dynamiquement les threads, car nous ne savions pas où se trouvait
la dernière pile allouée. Nous sommes donc passés à l'\textit{Action II.4}. Nous avons délégué
ces opérations à l'\textit{Address Space}. Nous avons créé un \textit{Bitmap} de taille égale au 
nombre de threads utilisateurs pouvant être créés par rapport à la taille de la pile allouée pour 
les stack des utilisateurs (\textit{UserStacksAreaSize/256}). Avant chaque création
de thread, nous vérifions si nous avons un espace de libre pour la stack et nous créons le thread que s'il y a de la place.

\subsection{Partie III}
Si un thread n'appelait pas \textit{ThreadExit}, il ne pouvait pas se détruire. Pour éviter cela, nous avons implémenté la terminaison 
automatique (mis à part pour le thread \textit{main}). Nous avons utilisé une autre méthode que celle donnée dans le TD.


\subsection{Partie IV}
Lorsque nous avons voulu remonter l'utilisation des sémaphores vers l'espace
utilisateur, nous avons beaucoup réfléchi à l'implémentation. Nous avons
rajouté quatre appels systèmes : \textit{SemaphoreCreate},
\textit{SemaphoreDelete}, \textit{P} et \textit{V}. Nous déléguons aux threads
d'un même espace d'adressage ont accès aux mêmes sémaphores. Pour cela, nous
avons un \textit{Bitmap} afin de savoir quels sémaphores sont créés et un
tableau de Sémaphores pour les stocker. L'utilisateur n'a réellement accès qu'à
l'indice de son sémaphore. Lorsque nous avons commencé à écrire nos fonctions P
et V dans \textit{addrspace.cc}, nous avions protégé les fonctions à l'aide
d'un mutex. Cela nous a mené à des deadlocks lorsque nous avons testé
l'implémentation avec le test producer-consumer. Nous avons donc retiré ce
mutex, en effet, les fonctions P et V de \textit{synch.cc} gèrent déjà la
concurrence des threads.

\subsection{Tests}
Lorsque nous avons fait nos tests, nous avons eu un problème avec notre
\textit{printf}. En effet, nous nous sommes rendus compte qu'il ne gérait pas
les boucles for. Nous n'avons pas réussi à expliquer le comportement de notre
fonction et avons donc choisi d'utiliser des \textit{PutString} et
\textit{PutInt} à la place.

\section{Limitations}
\subsection{Partie I}
Beaucoup de memory leak avec les threads : les fonctions de Cleanup ne
suppriment pas la stack. En effet, il y a bien une fonction
\textit{StackAllocate} mais pas de \textit{StackDeallocate}. Nous nous
retrouvons donc avec 73K octets non supprimés à la fin de l'exécution lorsque
nous utilisons des threads.

\subsection{Partie II}
Nous utilisons la variable \textit{UserStacksAreaSize} du fichier
\textit{addrspace.h} afin de créer les piles de nos threads. Nous avons gardé
la valeur de départ 1024, ce qui nous donne quatre piles de 256 octets.
L'utilisateur ne peut donc avoir que 4 threads en simultané. Si nous avons
besoin de plus, il suffit de modifier la variable. Nous avons gardé cette
valeur, car il était plus simple de se rendre compte de possibles erreurs avec
4 threads.

Lorsque nous n'avons pas de place pour créer un nouveau thread, on ne le crée pas et on renvoie
\textit{-1}. Il serait peut-être judicieux d'avoir une liste d'attente pour les threads non 
crées par manque de place. En effet, si nous voulons créer un thread, mais qu'il n'a pas de place,
il ne se créera pas et donc n'exécutera jamais son code.

\subsection{Partie IV}
Concernant les sémaphores utilisateurs, nous avons choisi d'avoir un bitmap de
taille 16, donc 16 sémaphores en simultané pour un même espace d'adressage.
Cette limite est modifiable dans le fichier \textit{addrspace.cc} à
\textit{MAX\_SEMAPHORES}, dans le cas où 16 serait une limite bien trop faible.

\section{Tests}

\section{Brouillon}

\end{document}